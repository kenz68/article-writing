\documentclass[12pt,a4paper]{report}
\usepackage{algorithm,algorithmic}
\usepackage{xcolor} % Load the xcolor package

\definecolor{myblue}{RGB}{52, 152, 219} % Define a custom blue color
\definecolor{lightgray}{RGB}{173, 173, 173}
\begin{document}
\section*{Object-Oriented Programming}
Object Oriented Programming is a paradigm in programming based on the concept of "objects". Objects contain data in the form of properties and methods.
It is the most popular style used in programming due to its modulatiry, code reusability and organisability.
In general naming convention, basic concepts of OOPS include:

\begin{enumerate}
    \item[\textbf{1.}]\textbf{Class:} A class is a blueprint of an object to be created.
    \item[\textbf{2.}]\textbf{Object:} Instance of class containing both data and methods.
    \item[\textbf{3.}]\textbf{Encapsulation:} Limiting access to properties, methods or variables in a class for code external to that class.
    \item[\textbf{4.}]\textbf{Inheritance:} It allows a class (commonly referred to as subclass) to inherit properties and methods from another class (commonly referred to as parent class).
    \item[\textbf{5.}]\textbf{Polymorphism:} Allowing objects of different classes to be treated as objects of a common superclass.
    \item[\textbf{6.}]\textbf{Abstraction:} Abstracting a basic concept of multiple classes in a single abstract class. This simplifies logic and makes code more readable.     
\end{enumerate}
We don't dive into the details and understanding of OOPS in this article. This article focuses on the implementation of these concepts in GO progamming language,
since unlike normal programming languages like Java, C++, etc, Go doesn't have the concepts of OOPS directly with common naming conventions.
For example, it does not have have a \textbf{\textit{class}} keyword.
\section*{OOP in GO}
Although GO does not have classes, it allow us to use OOP concepts using structs and interfaces. Let's see how we can use
OOP concepts in GO with the examples:
\section*{Classes, properties and methods}
We can use structs in GO to achieve the same functionality as class in
other programming languages. Structs can have methods and properties.
Let's say we are building a billing application in which we want to define a company:
    type Company struct \{\newline
    \indent\indent Id      string\newline
    \indent\indent Name    string\newline
    \indent\indent Country string\newline
  \}\newline
  \newline
  func newCompany(name string, country string) Company \{
    \newline\indent return Company\{
      \newline\indent\indent Id:      uuid.New().String(),
      \newline\indent\indent Name:    name,
      \newline\indent\indent Country: country,
      \newline\indent\}
      \newline\}
  \newline\newline
Go does not have the concept of constructors and classes, so we defined a
custom function to return the Compay. This would work as a constructor
for the Company. We initialize Id in this function.
Now to create an object of type Company, we will call \textit{newCompany} method
like this:
\begin{quote}
  \colorbox{lightgray}{var company Company = newCompany("MyCompany", "india")}
\end{quote}
\begin{quote}
  Note: Function overloading isn't supported in Go. So to create multiple
  implementations we would have to create multiple functions of different names.
\end{quote}
Futher, we want to save a company to the database. We can create a method of \textit{Company} struct,
for this purpose:
\begin{quote}
  func(company Company) saveToDatabase() \{\\
  \textbf{\textrightarrow}fmt.Println("Saving Company")\\
  \}
\end{quote}
\end{document}